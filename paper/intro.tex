\section{Introduction}

Obtaining real-time counts of occupants in each room/zone within a building has been one of the holy grails of building automation. Occupancy counts can be used as an input to a wide range of controls in buildings drastically improving energy efficiency. For example,  ventilation can be set in proportion to occupancy, air conditioning can be disabled or turned down for empty or sparsely populated areas, lights can be unobtrusively turned on/off without waving at a motion detector and plug-loads in empty areas could be monitored or turned off. These techniques can make a building's energy consumption proportional to the number of people in the building, whereas today most buildings operate wastefully in just two static modes: fully occupied or unoccupied. For buildings that are never completely empty, such as labs and graduate student offices, this static regime implies that the building continues to operate at night almost the same as during the day.

The major challenge is to obtain counts of occupants cheaply, reliably and without requiring any new actions from the occupants. Previous efforts have used motion detectors (PIR) with reed switches~\cite{Agarwal2010, Lu2010}, infrared (IR) beams~\cite{sun9}, video cameras~\cite{Erickson11}, RFID tags, thermal imaging and CO2 sensors. All of these approaches require the deployment of new hardware, hampering scalability, and vary in their accuracy and granularity. Motion detectors can only detect the presence of people, not their number. CO2 sensors approximate the number of people in an enclosed space, but cannot granularly count people in cubicle areas. IR beams are cheap and accurate but must be placed at all entrances and exits to an area. RFID requires occupants to carry special tags and requires large readers to be placed at all entrances. 

Observing that most building occupants on campus regularly carry smartphones and other wireless devices, we explore people counting using existing infrastructure. Our guiding design principles are that no new actions should be required of occupants,  counts are required at zone-level granularity, and deployment time and hardware required should be minimized. We apply a {\it passive localization} technique used in the wireless networking literature for rogue access point detection~\cite{Faria2006, Laurendeau2010}, to locate and count occupants. With this technique, 802.11 wireless infrastructure is used to overhear packets sent by client devices and measure the received signal strength (RSSI).

%Classification vs localization

%Our localization infrastructure differs from that of previous work in that it is not focused on precision, but rather tuned towards the application of loose-grained occupancy detection for the purpose of localized building actuation. By accurately determining the number of occupants in each of a series of predefined zones, we can adjust the lighting and heating in each zone appropriately, our infrastructure need only fulfill the level of granularity provided by the building actuation controls.

%While there are multiple ways to obtain that zone-level count, they hold other drawbacks and challenges.  Motion detectors provide no way for the user to set settings, and make getting a reliable per-zone count difficult.  As for RFID-equipped cards, those are expensive and require deployment of specialized equipment.  By contrast, this approach leverages existing deployed infrastructure and technology, minimizing cost and difficulty of deployment while achieving a high level of efficacy.

%A large body of previous work done in the area of localization relies on specialized hardware, dense hardware placement, generous sets of training data, or a combination thereof. The ubiquitous nature of 802.11 wireless networks presents an opportunity for leveraging existing wireless infrastructure with the addition of a small number of cheap and widely available devices to create a localization solution that does not require training data, instead relying on robust algorithms to make use of existing data.

%\begin{itemize}
%\item talk about advantages of passive localization over active localization -- if we don't require the occupants to do much, then we don't have to rely on them for accuracy
%\item mention some of the previous work here
%\item maybe talk about with BAS, we can easily incorporate the infrastructure with our pre-built actuation stuff
%\end{itemize}