\section{Introduction}
The application of accurate occupancy detection to energy-saving techniques in buildings provides a viable detour around the ``human factor.'' Personalized lighting and heating controls as well as intelligent scheduling of these systems are often ignored or overridden by occupants who simply do not want to expend the time or effort to learn or use such systems. Simple and robust occupancy detection offers an automated solution to utilizing the energy-saving techniques that may have already been implemented in a building.

Our localization infrastructure differs from that of previous work in that it is not focused on precision, but rather tuned towards the application of loose-grained occupancy detection for the purpose of localized building actuation. By accurately determining the number of occupants in each of a series of predefined zones, we can adjust the lighting and heating in each zone appropriately, our infrastructure need only fulfill the level of granularity provided by the building actuation controls. 

A large body of previous work done in the area of localization relies on specialized hardware, dense hardware placement, generous sets of training data, or a combination thereof. The ubiquitous nature of 802.11 wireless networks presents an opportunity for leveraging existing wireless infrastructure with the addition of a small number of cheap and widely available devices to create a localization solution that does not require training data, instead relying on robust algorithms to make use of existing data.